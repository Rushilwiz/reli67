\documentclass[
	letterpaper, % Paper size, use either a4paper or letterpaper
	10pt, % Default font size, can also use 11pt or 12pt, although this is not recommended
	unnumberedsections, % Comment to enable section numbering
	twoside, % Two side traditional mode where headers and footers change between odd and even pages, comment this option to make them fixed
]{LTJournalArticle}


\runninghead{The Code of Knowing} % A shortened article title to appear in the running head, leave this command empty for no running head

\footertext{\textit{PHIL 101, University of North Carolina} (2023)} % Text to appear in the footer, leave this command empty for no footer text

\setcounter{page}{1} % The page number of the first page, set this to a higher number if the article is to be part of an issue or larger work

%----------------------------------------------------------------------------------------
%	TITLE SECTION
%----------------------------------------------------------------------------------------

\title{From Scriptures to Society: A Dialogical Approach to Class Equality in Hinduism and Christianity} % Article title, use manual lines breaks (\\) to beautify the layout

% Authors are listed in a comma-separated list with superscript numbers indicating affiliations
% \thanks{} is used for any text that should be placed in a footnote on the first page, such as the corresponding author's email, journal acceptance dates, a copyright/license notice, keywords, etc
\author{%
	Rushil Umaretiya, Sara Simmons, Ananya Yenduri, Carolus Wang
}

% Affiliations are output in the \date{} command
\date{\footnotesize{The University of North Carolina at Chapel Hill}}
% Full-width abstract
\renewcommand{\maketitlehookd}{%
	
}

%----------------------------------------------------------------------------------------

\begin{document}

\maketitle % Output the title section

%----------------------------------------------------------------------------------------
%	ARTICLE CONTENTS
%----------------------------------------------------------------------------------------

\section{Introduction}
The differences between Hindu and Christian views on class equality give us a nuanced lens to view and tackle today’s issues with racial and economic systemic inequality in America. In American culture, which is heavily influenced by Christian ideals, there is a large focus on how one must put forth individual effort to break down class barriers. On the other hand, Hindu beliefs highlight how important it is for an individual to fulfill their duty to the community by keeping a strong social balance. When we view the modern political climate through this dual-ideology, it shows us that we might be able to create policies that not only support individual hard work, but also ensure that the greater society is able to ensure equality for all.

In the United States, individualism upholds a lasting tradition rooted within ideals of Christianity. While various authors delivered their sympathies for it, an individualistic approach also predestined the nation’s future of division between social classes through exercising its influence on political institutions. Through the individualists’ natural tendency of insisting on a government with a minimal presence in the social sphere, the aspect of government’s role in redistribution in alleviating social inequality is greatly impaired, and a conscious effort to address the issue is largely neglected.

Among the early pioneers of individualism, Henry David Thoreau ties together both freedom and equality in regards to Christianity and American Democracy. Thoreau specifically draws a comparison between democracy in the U.S. and Europe in an attempt to illustrate his fondness for a lifestyle of “wildness and freedom” (Thoreau 2017: 8) and to lay an emphasis for his vision of America as more preferable. As Thoreau enumerates, the vastness of the U.S : “wood or meadow or deserted pasture or hill” (Thoreau 2017:8) is a land of prospect and freedom which he is attracted to roam about. Whereas when Thoreau mentions Europe, which he refers to as the “east”, he considers it to be a place of stagnation and ossification due to densely constructed institutions, a place where he would never be willingly. Thoreau clearly states that democracy is successful in America because of the emphasis on individualism, which is encouraged by Christianity. In the end, he stressed that people of the states ought not adhere to the old Europe, for the future lies on the “Lethe of the Pacific” (Thoreau 2017 :8).

Hinduism, on the other hand, with its rich interwoven tales and legends, is able to offer us a unique perspective on the issues of class equality and systemic inequality. By examining fundamental principles of Hindu philosophy and social structure, we can gain new insight on how Hinduism approaches these issues of racial inequality and relate it to the current contemporary political ignorance behind a majority of issues faced by the greater samaj.

At the heart of Hindu society lies the concept of dharma, a multifaceted term that encompasses duty, righteousness, and moral correctness. Dharma dictates how each individual carries specific responsibilities and duties to their greater community based on their societal role, gender, and social class. This sense of duty extends beyond simple notions of tasks or service of the individual, but emphasizes the importance of maintaining a sense of balance and harmony within the larger community and society.

For the Newar people of Bhaktapur, Nepal, as illustrated by Steven Parish in Moral Knowing in a Hindu Sacred City, dharma is experienced simultaneously as religious, ethical, and social obligation. Parish says, “‘Society’ has priority over “self” in Newar moral thought, at least relative to the extreme individualism cherished by some in the Western world,” (Parish 1994:118). It is not only the personal or spiritual journey, but a collective responsibility to uphold moral and social order. The communal aspect of dharma fosters a sense of interconnectedness and mutual dependence on your neighbor. Newar society highlights the importance of each individual’s role in maintaining a societal balance. Not only is the act of practicing dharma used in order to form a samaj; it plays a significant role in sustaining ancient values: “With this sacrificial obligation did the gods offer the sacrifice. They were the first norms [dharma] of sacrifice,” (Rg. veda 10.90).

\section{Religious Perspectives on Class Equality}
Contrasting this idea of societal harmony, Hindu society has historically been organized into a rigid caste system, which categorizes individuals based on their birth and places them into certain social roles. While in the modern age this system has been federally abolished, its cultural and societal aftershocks still ring throughout Hindu society and it still lives in the minds of many Hindus today. Dumont writes when describing the caste system, “Think rather of the child, slowly brought to humanity by his upbringing in the family by the apprenticeship of language and moral judgment … which makes him share in the common patrimony” (Dumont 1970: 5) This system has been a tool of centuries of oppression with lower castes being restricted access to wealth, education, occupation, and livelihoods.

Even though this hierarchy was written into society through hymns for centuries, it is crucial to understand that Hindu scripture also strongly emphasizes the intrinsic equality of all souls, regardless of caste or social status. “The Brahman was his mouth, of both his arms was the Rājanya made. His thighs became the Vaiśya, from his feet the Śūdra was produced.” (Rg. veda 10.90). The Gita, a central Hindu text, teaches the eternal soul beyond birth and death, and that all souls are equal in the eyes of the divine. This spiritual equality serves as a stark counterpoint to the inequality at the core of the caste system, creating a complex and often contradictory view on class and equality.

With regards to Christianity and the ideal of individualism, extreme aspects of the religion were viewed as a threat to democracy. Hamilton argues that the Calvinist perspective affects the U.S. political system in which vigilance and distrust was prioritized during the devising process. Due to a negative view on human nature, the founders believed that people with power would exert every opportunity to expand and retain that power as a result of their inborn greed. Thus, when envisioning the future political system, the founders also set up guards in advance to prohibit centralization of power. A notable example would be the “division of power” into multiple branches and a further separation of the legislature as two entities to prevent the aggregation of power (Hamilton 2001:299).

Tocqueville is another author that had distinct views on Christianity and American politics. In Tocqueville’s notion of America, Christianity serves as a potent factor contributing to democracy, while also providing a theoretical basis for equality. Without equality in our laws and society, a cap is put on individualism. If humans do not have the same opportunities or resources available to them, they can not all succeed at their highest level. Christianity also encourages individualism. One of the founding ideas of Christianity is “God created man to his own image:  male and female he created them.” (Genesis 1:29). This shows that each person is their own individual and are unique while also being identical in essence. Toqueville says, “Poetry, eloquence, and memory, the graces of the mind, the fire of imagination, depth of thought, and all the gifts which Heaven scatters at aventure turned to the advantage of democracy; and even when they were in the possession of its adversaries, they still served its cause by throwing into bold relief the natural greatness of man” (Tocqueville 1994: 2). This quote clearly shows the connection between religion, democracy, and the individualistic nature of humans. This being said, at the time Toqueville wrote this article, freedom and equality did not necessarily exist for everyone in America. While most white, successful men (like Toquiville) had the freedom to pursue their own desires and be “free” within the law, slavery still existed and women were far from being considered equal to their male counterparts. 
\section{Modern Political Implications}

Modern America has come a long way from its inequality of the past. However, it is still hard to genuinely say that all Americans are equal. According to a 2016 study conducted by Pew Research Center,  43\% of non-white Americans agreed with the statement that “our country will not make the changes needed to give blacks equal rights with whites” (Pew 2016:1). In addition to this, a study done by the American Association of University Women states that, “Women working full time in the U.S. are still paid just 84 cents to every dollar earned by men — and the consequences of this gap affect women throughout their lives. The pay gap even follows women into retirement: As a result of lower lifetime earnings, they receive less in Social Security and pensions. In terms of overall retirement income, women have only 70\% of what men do” (AAUW 2020 : 1). This study shows that while women may now be viewed as equals to their male counterparts, most still economically are not. 

The Hindu emphasis on community and interconnectedness also provides us a valuable perspective in addressing systemic inequality. Unlike the individualistic approach in Western, Christian-influenced societies, Hinduism encourages individuals to see themselves as a part of the larger whole, with duties and responsibility to the community. Community, a term which encompasses those of higher and lower class and caste than the individual, is something that the soul is bound to; providing for such gives one’s being value.

Addressing the persistent issues of racial and economic disparity in the United States necessitates a practical application of Hindu concepts such as dharma (duty/righteousness) and samaj (community/society). By embedding the principles of dharma in our societal structures, we can promote a sense of shared responsibility among individuals to actively participate in dismantling systemic inequalities. This involves creating and enforcing policies that ensure equal opportunities in education and healthcare, regardless of one's racial background, thereby addressing some of the root causes of disparity.

\section{Conclusion}
By examining the distinct perspectives of Hinduism and Christianity on class equality and systemic inequality, this paper unveils a multifaceted understanding of these deep-seated issues within American society. Hinduism, with its doctrine of dharma, encourages a sense of duty to the community and promotes social balance, urging individuals to view themselves as integral parts of a larger whole. Christianity’s influence on American values underscores the power of individual effort and personal accountability, essential for overcoming systemic barriers. Yet, isolated, each perspective offers an incomplete solution; the American emphasis on individualism can lead to neglect of systemic issues, while a solely community-focused approach may overlook the transformative potential of individual agency.

Synthesizing the strengths of both Hindu and Christian perspectives, we can foster a balanced view that promotes individual empowerment within a framework of social responsibility. This harmonious blend advocates for policies that support personal diligence while concurrently addressing structural inequalities, aiming for a society that values both individual achievement and communal well-being. In navigating the challenges of racial and economic inequality, this dual perspective serves as a comprehensive lens, guiding the pursuit of a more equitable and just society.


\end{document}
